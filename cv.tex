%!TEX TS-program = xelatex
%!TEX encoding = UTF-8 Unicode
% Awesome CV LaTeX Template for CV/Resume
%
% This template has been downloaded from:
% https://github.com/posquit0/Awesome-CV
%
% Author:
% Claud D. Park <posquit0.bj@gmail.com>
% http://www.posquit0.com
%
% Template license:
% CC BY-SA 4.0 (https://creativecommons.org/licenses/by-sa/4.0/)
%


%-------------------------------------------------------------------------------
% CONFIGURATIONS
%-------------------------------------------------------------------------------
% A4 paper size by default, use 'letterpaper' for US letter
\documentclass[11pt, a4paper]{awesome-cv}

% Configure page margins with geometry
\geometry{left=1.4cm, top=.8cm, right=1.4cm, bottom=1.4cm, footskip=.5cm}

% Specify the location of the included fonts
\fontdir[fonts]

% Color for highlights
% Awesome Colors: awesome-emerald, awesome-skyblue, awesome-red, awesome-pink, awesome-orange
%                 awesome-nephritis, awesome-concrete, awesome-darknight
\colorlet{awesome}{awesome-red}
% Uncomment if you would like to specify your own color
% \definecolor{awesome}{HTML}{CA63A8}

% Colors for text
% Uncomment if you would like to specify your own color
% \definecolor{darktext}{HTML}{414141}
% \definecolor{text}{HTML}{333333}
% \definecolor{graytext}{HTML}{5D5D5D}
% \definecolor{lighttext}{HTML}{999999}

% Set false if you don't want to highlight section with awesome color
\setbool{acvSectionColorHighlight}{true}
%\usepackage[libertine]{newtxmath}
% If you would like to change the social information separator from a pipe (|) to something else
\renewcommand{\acvHeaderSocialSep}{\quad\textbar\quad}


%-------------------------------------------------------------------------------
%	PERSONAL INFORMATION
%	Comment any of the lines below if they are not required
%-------------------------------------------------------------------------------
% Available options: circle|rectangle,edge/noedge,left/right
% \photo{./examples/profile.png}

\name{Yuki}{Kume}
\position{University Student{\enskip\cdotp\enskip}Independent Developer}
%\address{42-8, Bangbae-ro 15-gil, Seocho-gu, Seoul, 00681, Rep. of KOREA}
%\mobile{(+82) 10-9030-1843}
\email{yuki0003@student.monash.edu}
%\homepage{unciabit.com}
\github{UnciaBit}
%\github{WrenHainsworth}
\linkedin{YukiKume}
% \gitlab{gitlab-id}
% \stackoverflow{SO-id}{SO-name}
% \twitter{@twit}
% \skype{skype-id}
% \reddit{reddit-id}
% \medium{madium-id}
% \googlescholar{googlescholar-id}{name-to-display}
%% \firstname and \lastname will be used
% \googlescholar{googlescholar-id}{}
% \extrainfo{extra informations}

\quote{Student developer casually developing for the past 7 years in software and web with focus in network \& security - a curiosity monster who loves experimenting with new technologies and implementing them in practice and love challenging self to limits to expand the area of expertise and deepen the skills. Current topic of interest and aspiring research topic in unity of AI, HPC, and (network/software) security}


%-------------------------------------------------------------------------------
\begin{document}

% Print the header with above personal informations
% Give optional argument to change alignment(C: center, L: left, R: right)
  \makecvheader[R]

% Print the footer with 3 arguments(<left>, <center>, <right>)
% Leave any of these blank if they are not needed
  \makecvfooter
  {\today}
  {Yuki Kume}
  {\thepage}


%-------------------------------------------------------------------------------
%	CV/RESUME CONTENT
%	Each section is imported separately, open each file in turn to modify content
%-------------------------------------------------------------------------------
  %-------------------------------------------------------------------------------
%	SECTION TITLE
%-------------------------------------------------------------------------------
\cvsection{Education}


%-------------------------------------------------------------------------------
%	CONTENT
%-------------------------------------------------------------------------------
\begin{cventries}


    \cventry
    {Monash University}
    {Bachelor of Computer Science Advanced (Honours)}
    {Melbourne, Australia}
    {Feb. 2022 - Present}
    {
        \begin{cvitems}
            \item {Major in Computer Networks and Security}
            \item {Member Training Officer at MonSec (Monash University Cybersecurity Club)}
            \item {GPA: 3.75/4.0 | WAM: 87}
        \end{cvitems}
    }


%---------------------------------------------------------
  \cventry
    {Dulwich College Suzhou} % Degree
    {International Baccalaureate} % Institution
    {Suzhou, China} % Location
    {Sep. 2013 - Jun. 2021} % Date(s)
    {
      \begin{cvitems} % Description(s) bullet points
        \item {Media Prefect (2020-2021)}
      \end{cvitems}
    }

%---------------------------------------------------------
\end{cventries}

  %-------------------------------------------------------------------------------
%	SECTION TITLE
%-------------------------------------------------------------------------------
\cvsection{Skills}


%-------------------------------------------------------------------------------
%	CONTENT
%-------------------------------------------------------------------------------
\begin{cvskills}

%---------------------------------------------------------
  \cvskill
    {DevOps} % Category
    {Docker, Kubernetes, Git, Prometheus, Travis CI, GitHub CI/CD/Actions, Ansible} % Skills

%---------------------------------------------------------
  \cvskill
    {Programming} % Category
    {Python, C++, JavaScript/TypeScript, C\#, Rust, Java, PHP, Node.js | DB: MySQL, PostgreSQL, SQLite, MongoDB, Redis} % Skills
    
%   \cvskill
%    {Specialised Software} % Category
%    {IDA/Ghidra, Wireshark, Burp Suite, gdb, Metasploit, Nmap} % Skills
    
%---------------------------------------------------------
  \cvskill
    {Operating Systems} % Category
    {Fedora, RHEL 7/8, Kali Linux, Ubuntu, macOS, Windows 10, CentOS, OpenSUSE} % Skills
%
%  \cvskill
%   {Soft Skills}
%   {Communication, Teamwork, Self-learning, Time Management, Problem Solving, Leadership, Flexibility/Adaptability}

%---------------------------------------------------------
\end{cvskills}
\vspace{-3.0mm}

  %-------------------------------------------------------------------------------
%	SECTION TITLE
%-------------------------------------------------------------------------------
\cvsection{Experience}


%-------------------------------------------------------------------------------
%	CONTENT
%-------------------------------------------------------------------------------
\begin{cventries}

  \cventry
  {Monash University, eSolutions, Infrastructure Platforms} % Organization
  {Systems Officer} % Job title
  {Clayton, Australia} % Location
  {Apr. 2022 - Present} % Date(s)
  {
    \begin{cvitems} % Description(s) of tasks/responsibilities
      \item {Assist in migration of the university's infrastructure automation from Puppet to Ansible}
    \end{cvitems}
  }

%---------------------------------------------------------
  \cventry
    {Monash University cybersecurity research group \& The Legion of the Bouncy Castle Inc.} % Organization
    {Research Assistant (C\# Programmer)} % Job title
    {Remote, Australia} % Location
    {Apr. 2022 - Present} % Date(s)
    {
      \begin{cvitems} % Description(s) of tasks/responsibilities
        \item {Developing a .NET cryptography library for a post-quantum lattice-based public-key cryptography algorithm to add to the Bouncy Castle library}
        \item {Currently assigned the C\# implementation of NIST PQC Round 3 NTRU Prime/NTRU algorithms from C reference implementations (solo)}
        \item {Using C\# + .NET Framework Standard 2.0}
%        \item {Initial implementation released to public through \href{https://github.com/bcgit/bc-csharp/tree/master/crypto/src/pqc/crypto/ntruprime}{\color{red}Bouncy Castle GitHub Repository (link)} on 22/07/22}
      \end{cvitems}
    }

%--------------------------------------------------------

%---------------------------------------------------------
\end{cventries}

  %-------------------------------------------------------------------------------
%	SECTION TITLE
%-------------------------------------------------------------------------------
\cvsection{Research Projects}


%-------------------------------------------------------------------------------
%	CONTENT
%-------------------------------------------------------------------------------
\begin{cventries}

%--------------------------------------------------------

   \cventry
   {Supervisor: Dr. Xingliang Yuan | Faculty: Faculty of Information Technology} % Organization
   {Privacy-Preserving Server-Side Content Moderation over E2EE Communication} % Job title
   {Monash University} % Location
   {Jun. 2022 - Present} % Date(s)
   {
      \begin{cvitems} % Description(s) of tasks/responsibilities
         \item {Literature review on: stateful network functions and protocols, TCP and TCP state machine, finite state machine, packet inspection}
      \end{cvitems}
   }

%---------------------------------------------------------
\end{cventries}
\vspace{-3.0mm}

  %-------------------------------------------------------------------------------
%	SECTION TITLE
%-------------------------------------------------------------------------------
\cvsection{Leadership}


%-------------------------------------------------------------------------------
%	CONTENT
%-------------------------------------------------------------------------------
\begin{cventries}

    \cventry
    {Google \& Monash University} % Organization
    {Google Student Developer Clubs (GDSC) Lead} % Job title
    {Australia} % Location
    {Jun. 2022 - Present} % Date(s)
    {
        \begin{cvitems} % Description(s) of tasks/responsibilities
            \item {Establish GDSC Monash University Chapter by partnering with IT Faculty's official club and recruiting core team members}
            \item {Administrate all logistics and workflow of the club. Establishing processes and procedures for the club operations}
            \item {Design and present technical workshops (GCP and Android) to present to community members}
        \end{cvitems}
    }

    \cventry
    {Monash Cyber Security Club (MonSec)} % Organization
    {Member Training Officer} % Job title
    {Australia} % Location
    {Jun. 2022 - Present} % Date(s)
    {
        \begin{cvitems} % Description(s) of tasks/responsibilities
            \item {Designed and ran a workshop on XXE Web Exploitation and demonstration of HackTheBox challenges}
            \item {Inviting guest speakers to present topics in cyber security}
        \end{cvitems}
    }


    \cventry
    {Dulwich College Suzhou} % Organization
    {Media Prefect} % Job title
    {Australia} % Location
    {Jun. 2022 - Present} % Date(s)
    {
        \begin{cvitems} % Description(s) of tasks/responsibilities
            \item {Managing the media team, directed and involved in the coverage of school events and producing promotional/coverage videos}
            \item {Responsible for equipment purchase decisions over \$AUD 30,000}
            \item {Working closely with the marketing department to ensure the school's brand is represented consistently}
        \end{cvitems}
    }


%--------------------------------------------------------

%---------------------------------------------------------
\end{cventries}

%%-------------------------------------------------------------------------------
%	SECTION TITLE
%-------------------------------------------------------------------------------
\cvsection{Activities}


%-------------------------------------------------------------------------------
%	CONTENT
%-------------------------------------------------------------------------------
\begin{cvhonors}

%---------------------------------------------------------
   \cvhonor
   {Google Student Developer Clubs (GDSC) Lead}
   {Lead core team members, design workshops, workflow opt.}
   {Google}
   {Jun 2022}

   \cvhonor
   {Workshop Leader/Designer}
   {Design and present MonSec XXE Web Exploit Workshop}
   {MonSec}
   {Apr 2022}

%   \cvhonor
%   {Plaid CTF 2022 (Solo)}
%   {Web Exploitation}
%   {Plaid CTF, Online}
%   {2022 Apr}
%
%
%   \cvhonor
%   {RITSEC CTF 2022 (Team)}
%   {Web Exploitation, Binary Exploitation, Cryptography}
%   {RITSEC CTF, Online}
%   {2022 Apr}


%   \cvhonor
%   {UMassCTF 2022 (Team)}
%   {Web Exploitation}
%   {UMass CTF, Online}
%   {2022 Apr}
%
%
%   \cvhonor
%   {MonSec CTF (Team)}
%   {Web Exploitation, Binary Exploitation, Cryptography, Forensics, OSINT}
%   {MonSec, Monash University, Australia}
%   {2022 Mar}
%
%
%   \cvhonor
%   {picoCTF (Solo)}
%   {Cryptography, Reverse Engineering, Binary Exploitation, Web Exploitation, Forensics}
%   {picoCTF, Online}
%   {2022 Mar}
%
   \cvhonor
   {UNIHACK Hackathon Participant} % Position
   {Core development of project: Decentralised Social Media} % Committee
   {Australia} % Location
   {Feb 2022} % Date(s)

   \cvhonor
   {Media Prefect} % Position
   {Manage media team, communicate with marketing dept., equipment purchase decision} % Committee
   {China} % Location
   {2020-2021} % Date(s)

   \cvhonor
   {Hackathon Staff + Workshop leader/teacher} % Position
   {Design \& run Python workshop, manage event logistics} % Committee
   {China} % Location
   {2018} % Date(s)

   \cvhonor
   {DCS Hackathon Participant} % Position
   {Development of project: Delivery tracking application} % Committee
   {China} % Location
   {2018} % Date(s)

   \cvhonor
   {Elegant365 Web dev. + Server admin} % Position
   {Development of website and administration of its infrastructure} % Committee
   {China} % Location
   {2017-2020} % Date(s)

%   \cvhonor
%   {Participant} % Position
%   {GTE entrepreneurship camp/competition} % Committee
%   {Japan} % Location
%   {2017} % Date(s)

%---------------------------------------------------------
\end{cvhonors}

  %-------------------------------------------------------------------------------
%	SECTION TITLE
%-------------------------------------------------------------------------------
\cvsection{Honors \& Awards}


%-------------------------------------------------------------------------------
%	SUBSECTION TITLE
%-------------------------------------------------------------------------------
\cvsubsection{International}


%-------------------------------------------------------------------------------
%	CONTENT
%-------------------------------------------------------------------------------
\begin{cvhonors}

%---------------------------------------------------------
  \cvhonor
    {Finalist} % Award
    {DEFCON 26th CTF Hacking Competition World Final} % Event
    {Las Vegas, U.S.A} % Location
    {2018} % Date(s)

%---------------------------------------------------------
  \cvhonor
    {Finalist} % Award
    {DEFCON 25th CTF Hacking Competition World Final} % Event
    {Las Vegas, U.S.A} % Location
    {2017} % Date(s)

%---------------------------------------------------------
  \cvhonor
    {Finalist} % Award
    {DEFCON 22nd CTF Hacking Competition World Final} % Event
    {Las Vegas, U.S.A} % Location
    {2014} % Date(s)

%---------------------------------------------------------
  \cvhonor
    {Finalist} % Award
    {DEFCON 21st CTF Hacking Competition World Final} % Event
    {Las Vegas, U.S.A} % Location
    {2013} % Date(s)

%---------------------------------------------------------
  \cvhonor
    {Finalist} % Award
    {DEFCON 19th CTF Hacking Competition World Final} % Event
    {Las Vegas, U.S.A} % Location
    {2011} % Date(s)

%---------------------------------------------------------
  \cvhonor
    {6th Place} % Award
    {SECUINSIDE Hacking Competition World Final} % Event
    {Seoul, S.Korea} % Location
    {2012} % Date(s)

%---------------------------------------------------------
\end{cvhonors}


%-------------------------------------------------------------------------------
%	SUBSECTION TITLE
%-------------------------------------------------------------------------------
\cvsubsection{Domestic}


%-------------------------------------------------------------------------------
%	CONTENT
%-------------------------------------------------------------------------------
\begin{cvhonors}

%---------------------------------------------------------
  \cvhonor
    {3rd Place} % Award
    {WITHCON Hacking Competition Final} % Event
    {Seoul, S.Korea} % Location
    {2015} % Date(s)

%---------------------------------------------------------
  \cvhonor
    {Silver Prize} % Award
    {KISA HDCON Hacking Competition Final} % Event
    {Seoul, S.Korea} % Location
    {2017} % Date(s)

%---------------------------------------------------------
  \cvhonor
    {Silver Prize} % Award
    {KISA HDCON Hacking Competition Final} % Event
    {Seoul, S.Korea} % Location
    {2013} % Date(s)

%---------------------------------------------------------
  \cvhonor
    {2nd Award} % Award
    {HUST Hacking Festival} % Event
    {S.Korea} % Location
    {2013} % Date(s)

%---------------------------------------------------------
  \cvhonor
    {3rd Award} % Award
    {HUST Hacking Festival} % Event
    {S.Korea} % Location
    {2010} % Date(s)

%---------------------------------------------------------
  \cvhonor
    {3rd Award} % Award
    {Holyshield 3rd Hacking Festival} % Event
    {S.Korea} % Location
    {2012} % Date(s)

%---------------------------------------------------------
  \cvhonor
    {2nd Award} % Award
    {Holyshield 3rd Hacking Festival} % Event
    {S.Korea} % Location
    {2011} % Date(s)

%---------------------------------------------------------
  \cvhonor
    {5th Place} % Award
    {PADOCON Hacking Competition Final} % Event
    {Seoul, S.Korea} % Location
    {2011} % Date(s)

%---------------------------------------------------------
\end{cvhonors}

  %-------------------------------------------------------------------------------
%	SECTION TITLE
%-------------------------------------------------------------------------------

% New line

%\pagebreak

\cvsection{Development Projects}


%-------------------------------------------------------------------------------
%	CONTENT
%-------------------------------------------------------------------------------
\begin{cventries}

  \cventry
  {C\#, .NET, C}
  {NTRU and NTRU Prime .NET Post Quantum Cryptography Library}
  {\href{https://github.com/bcgit/bc-csharp/tree/master/crypto/src/pqc/crypto/ntruprime}{GitHub (NTRU Prime)}, \href{https://github.com/bcgit/bc-csharp/tree/master/crypto/src/pqc/crypto/ntru}{GitHub (NTRU)}} % Location
  {Apr 2022 - Present}
  {
    \begin{cvitems}
      \item {Developed a post-quantum lattice-based public-key cryptography library for .NET Framework Standard 2.0}
      \item {Implemented algorithms are NIST PQC Round 3's NTRU Prime and NTRU algorithms}
      \item {Open-sourced through Bouncy Castle's GitHub repository}
      \item {Part of a research project with Monash University's cybersecurity research group and The Legion of the Bouncy Castle Inc.}
    \end{cvitems}
  }

  \cventry
  {Python, Flask, PyTorch, ONNX, TypeScript, NodeJS, GCP, Docker, VSCode API}
  {AI Bug Hunter - ML-based Security Vulnerability Detection Tool}
  {Under Development} % Location
  {Jun 2022 - Present}
  {
    \begin{cvitems}
      \item {VSCode Extension that detects vulnerable functions in real time, then provide analytics on the vulnerable line. It is part of a research project}
      \item {Using PyTorch AI Model built by a PhD Student at Monash University, and working closely with him during development}
      \item {Additionally developed Docker and Flask script to provide API for model inference. Using GCP Vertex AI to investigate cloud inference option}
      \item {Demoed at Monash University Open Day at Faculty of Information Technology, Department of Software Systems and Cybersecurity}
      \item {Pending submission to ICSE 2022}
    \end{cvitems}
  }

%  \cventry
%  {C\#, .NET Framework Standard 2.0}
%  {NTRU/NTRU Prime Bouncy Castle Library - Post Quantum Cryptography Library}
%  {Under Development} % Location
%  {Apr. 2022 - PRESENT}
%  {
%    \begin{cvitems}
%      \item {Client: Monash University + The Legion of the Bouncy Castle}
%      \item {C\# Library implementing NIST PQC Round 3 NTRU/NTRU Prime lattice-based post quantum cryptography algorithm}
%    \end{cvitems}
%  }

  \cventry
  {C++}
  {finman - Command Line Personal Finance Manager}
  {\href{https://github.com/UnciaBit/finman}{GitHub}} % Location
  {Jan 2022}
  {
    \begin{cvitems}
      \item {Command-line finance manager built using C++}
      \item {Using SQLite as backend database}
    \end{cvitems}
  }

  \cventry
  {C}
  {Sequence Analyser - DNA/RNA Sequence Analysis Tool}
  {\href{https://github.com/UnciaBit/Sequence-Analysis}{GitHub}} % Location
  {Jan 2022}
  {
    \begin{cvitems}
      \item {Simple and fast Sequence analysis tool (DNA/RNA) based on Aho-Corasick algorithm}
      \item {Inputs a sequence and a pattern, outputs the number of occurrences of the pattern in the sequence, and it's position}
    \end{cvitems}
  }



  \cventry
  {NodeJS, Javascript, Redis}
  {UnciaNet - Custom implementation of a distributed blockchain network}
  {\href{https://github.com/UnciaBit/UnciaNet-FE}{GitHub}} % Location
  {Dec 2021 - Present}
  {
    \begin{cvitems}
      \item {Blockchain network built from scratch with NodeJS and Redis backend}
      \item {NodeJS REST API made for creating transactions, Redis Pub/Sub for announcing blocks}
      \item {Currently under development to transform into blockchain visualisation/learning tool}
    \end{cvitems}
  }


%---------------------------------------------------------
  \cventry
    {Rust} 
    {KeySort - Command Line Utility for Sorting Files}
    {\href{https://github.com/UnciaBit/KeySort-cli}{GitHub}, \href{https://crates.io/crates/keysort}{Cargo}} % Location
    {Nov 2021 - Present}
    {
      \begin{cvitems}
        \item {Small command-line utility to sort files and folders into character assigned folders}
        \item {Build with Rust with efficiency and speed in consideration}
      \end{cvitems}
    }

%---------------------------------------------------------
  \cventry
    {PHP, HTML, CSS, Javascript, MySQL} 
    {EqManage - Equipment Management System}
    {\href{https://github.com/YukiKume/EqManage}{GitHub}} % Location
    {Oct 2019 - Jan. 2021}
    {
      \begin{cvitems}
        \item {Client: Dulwich College Suzhou}
        \item {Equipment (Renting) management system}
        \item {Users can rent equipment, set renting period + message. Admins can approve or deny the request. Includes dashboard, overdue tracking etc...}
      \end{cvitems}
    }

  \cventry
  {POEdit, PHP, CSS, Javascript, HTML}
  {cocoon - WordPress Theme}
  {\href{https://github.com/WrenHainsworth/cocoon}{GitHub}} % Location
  {Oct 2021 - Present}
  {
    \begin{cvitems}
      \item {Pull request contribution to an OpenSource project: yhira/cocoon}
      \item {SEO optimised WordPress theme}
      \item {Mainly contributed the translation that was then pushed to the official branch in the main repository}
    \end{cvitems}
  }

  \cventry
  {C\# (Unity), Cinema 4D}
  {Tower defense game}
  {} % Location
  {Jan 2018}
  {
    \begin{cvitems}
      \item {Tower Defense Game built using Unity}
      \item {Single player game, where the player defends endpoint from incoming enemies. All elements including the tower design were solo developed}
    \end{cvitems}
  }

%  \cventry
%  {PHP, MySQL, HTML, CSS, Javascript}
%  {Chatbox}
%  {} % Location
%  {Jul. 2016}
%  {
%    \begin{cvitems}
%      \item {Multi-user chatroom, with online status and chat history. Similar to Discord}
%    \end{cvitems}
%  }

%---------------------------------------------------------
\end{cventries}



%%-------------------------------------------------------------------------------
%	SECTION TITLE
%-------------------------------------------------------------------------------
\cvsection{Presentation}


%-------------------------------------------------------------------------------
%	CONTENT
%-------------------------------------------------------------------------------
\begin{cventries}

%---------------------------------------------------------
  \cventry
    {Presenter for <Hosting Web Application for Free utilizing GitHub, Netlify and CloudFlare>} % Role
    {DevFest Seoul by Google Developer Group Korea} % Event
    {Seoul, S.Korea} % Location
    {Nov. 2017} % Date(s)
    {
      \begin{cvitems} % Description(s)
        \item {Introduced the history of web technology and the JAM stack which is for the modern web application development.}
        \item {Introduced how to freely host the web application with high performance utilizing global CDN services.}
      \end{cvitems}
    }

%---------------------------------------------------------
  \cventry
    {Presenter for <DEFCON 20th : The way to go to Las Vegas>} % Role
    {6th CodeEngn (Reverse Engineering Conference)} % Event
    {Seoul, S.Korea} % Location
    {Jul. 2012} % Date(s)
    {
      \begin{cvitems} % Description(s)
        \item {Introduced CTF(Capture the Flag) hacking competition and advanced techniques and strategy for CTF}
      \end{cvitems}
    }

%---------------------------------------------------------
  \cventry
    {Presenter for <Metasploit 101>} % Role
    {6th Hacking Camp - S.Korea} % Event
    {S.Korea} % Location
    {Sep. 2012} % Date(s)
    {
      \begin{cvitems} % Description(s)
        \item {Introduced basic procedure for penetration testing and how to use Metasploit}
      \end{cvitems}
    }

%---------------------------------------------------------
\end{cventries}



  %-------------------------------------------------------------------------------
%	SECTION TITLE
%-------------------------------------------------------------------------------
\cvsection{Certifications}


%-------------------------------------------------------------------------------
%	CONTENT
%-------------------------------------------------------------------------------
\begin{cvhonors}

%---------------------------------------------------------
 \cvhonor{Neural Networks and Deep Learning}{No expiration}{\href{https://www.coursera.org/account/accomplishments/certificate/LD82Y2RMBBPW}{Coursera}}{2022 Jun}
 
 \cvhonor
 {Cybersecurity Fundamentals} % Position
 {No expiration date} % Expiration date
 {IBM SkillsBuild} % Location
 {2022 Feb} % Date(s)

 \cvhonor
 {NSE 1 Network Security Associate}
 {Expires Feb 2024}
 {Fortinet}
 {2022 Feb}

%---------------------------------------------------------
\end{cvhonors}

  %-------------------------------------------------------------------------------
%	SECTION TITLE
%-------------------------------------------------------------------------------
\cvsection{Personal Interests}


%-------------------------------------------------------------------------------
%	CONTENT
%-------------------------------------------------------------------------------
\begin{cventries}

%---------------------------------------------------------
  \cventry
    {Docker, Kubernetes, Server Administration}
    {Personal Server Administration}
    {Server}
    {2018 - Present}
    {
      \begin{cvitems}
        \item{Hosting services on both VPS (Vultr \& Hetzner) and personal server using Docker and orchestrated using Kubernetes}
        \item{Reverse proxy, SSL certificate, intrusion detection, access control, firewall and other basic networking security/configuration applied}
        \item{Current VPS Deployment: Ubuntu LTS 20.04. Current Local Deployment: CentOS. Also have experience RHEL}
      \end{cvitems}
    }

%  \cventry
%    {Web exploitation, Binary exploitation, Exploit Development, Reverse Engineering}
%    {HackTheBox and TryHackMe | Penetration Lab Boxes}
%    {Cybersecurity}
%    {2022 - PRESENT}
%    {
%      \begin{cvitems}
%        \item{Solving challenges on HackTheBox and TryHackMe}
%        \item{Main categories: Web exploitation, reverse engineering, pwning/binary exploitation, and cryptography}
%      \end{cvitems}
%    }


  \cventry
    {Penetration Testing: Web exploitation, Binary exploitation, Exploit Development, Reverse Engineering}
    {Cybersecurity}
    {Cybersecurity}
    {2022 - Present}
    {
      \begin{cvitems}
        \item{Solving challenges on HackTheBox}
        \item{Main categories: Web exploitation, reverse engineering, pwning/binary exploitation, and cryptography}
        \item{Attended CTFs: picoCTF 2022 (Solo), MonSec CTF 2022 (Team), UMassCTF 2022 (Team), RITSEC CTF 2022 (Team), Google CTF 2022 (Team)}
        \item{Attended SANS "FOR509: Enterprise Cloud Forensics and Incident Response" training workshop at Australian Cyber Conference 2022}
      \end{cvitems}
    }

%---------------------------------------------------------
\end{cventries}

%-------------------------------------------------------------------------------
\end{document}
